% !TEX encoding = IsoLatin

% La bibliografia, da inserirsi solo se ci sono state citazioni.
% In questo caso ricordarsi che bisogna sempre elaborare due volte il file .TEX
% perch� la prima volta viene generata la bibliografia mentre la seconda volta viene inclusa

% NOTA: citare il DOI non � obbligatorio ma MOLTO desiderabile
% NOTE: inserting the DOI is not compulsory bur STRONGLY recommended whenever it exists

\begin{thebibliography}{9} % se ci sono meno di 10 citazioni
%\begin{thebibliography}{99} % se ci sono da 10 a 99 citazioni
%\begin{thebibliography}{999} % se ci sono da 100 a 999 citazioni

\begin{comment}
% esempio citazione articolo a congresso
% example: reference to a conference paper
\bibitem{psisec}
% autori - authors
I.Enrici, M.Ancilli, A.Lioy,
% titolo articolo - article title
``A psychological approach to information technology security'',
% nome del congresso - conference name
HSI-2010: 3rd Int. Conf. on Human System Interactions,
% luogo (stato) e data del congresso
% town (country) and date of the conference
Rzesz�w (Poland), May 13-15, 2010,
% pagine dell'articolo - article pages
pp.\ 459-466,
% DOI
\doi{10.1109/HSI.2010.5514528}

% esempio citazione articolo su rivista
% example: reference to a journal/magazine article
\bibitem{tpa}
% autori- authors
G.Cabiddu, E.Cesena, R.Sassu, D.Vernizzi, G.Ramunno, A.Lioy,
% titolo dell'articolo -  article title
``Trusted Platform Agent'',
% nome della rivista - name of the journal
IEEE Software,
% volume e numero della rivista (alcune riviste non ce l'hanno)
% volume and issue number (some journals don't have it)
Vol.\ 28, No.\ 2,
% mese e anno di pubblicazione della rivista
% month and year when paper appeared in the journal
March-April 2011,
% pagine dell'articolo  - article pages
pp.\ 35-41,
% DOI
\doi{10.1109/MS.2010.160}


% esempio citazione capitolo di un libro fatto come collezione di contributi da autori diversi
% example: reference to the chapter of a book which is a collection of chapters from different authors
\bibitem{tc}
A.Lioy, G.Ramunno, % autori del capitolo
``Trusted Computing'' % titolo del capitolo
nel libro % in the book
``Handbook of Information and Communication Security'' % titolo del libro
a cura di % edited by
P.Stavroulakis, M.Stamp, % nomi dei curatori
Springer, % nome editore
2010, % anno di pubblicazione
pp.\ 697-717, % pagine del capitolo
\doi{10.1007/978-3-642-04117-4_32}

% esempio citazione pagina web di un progetto
% example: reference to the web page pof a project
\bibitem{openssl}
% nome del progetto - name of the project
The OpenSSL project,
 % URI della pagina web - URI of the web page
\url{http://www.openssl.org/}

% esempio citazione RFC
% example: reference to a RFC
\bibitem{tls12}
T.Dierks, E.Rescorla,
``The Transport Layer Security (TLS) Protocol Version 1.2'',
\rfc{5246}, August 2008,
\doi{10.17487/RFC5246}

% esempio: citazione libro
% example: reference to a book
\bibitem{seceng}
Ross J. Anderson,
``Security engineering'',
Wiley, 2008,
ISBN: 978-0-470-06852-6
\end{comment}

% esempio citazione pagina web di un progetto
\bibitem{fips81}
% nome del progetto - name of the project
FIPS PUB 81,
 % URI della pagina web - URI of the web page
\url{https://csrc.nist.gov/csrc/media/publications/fips/81/archive/1980-12-02/documents/fips81.pdf}

\bibitem{800-38G}
SP 800-38G, 
\doi{http://dx.doi.org/10.6028/NIST.SP.800-38G}

\bibitem{fips74}
FIPS PUB 74,
\url{https://nvlpubs.nist.gov/nistpubs/Legacy/FIPS/fipspub74.pdf}

\bibitem{BlackRog}
J. Black, P. Rogaway, "Ciphers with Arbitrary Finite Domains",
\url{https://www.cs.ucdavis.edu/~rogaway/papers/subset.pdf}

\end{thebibliography}
