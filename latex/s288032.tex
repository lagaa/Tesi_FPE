% !TEX encoding = IsoLatin

% La riga soprastante serve per configurare gli editor TeXShop, TeXWorks
% e TeXstudio per gestire questo file con la codifica IsoLatin o Latin 1
% o ISO 8859-1.

% per commentare una riga mettere % al suo inizio
% per s-commentare una riga (ossia attivarla) togliere il % al suo inizio
%
\documentclass[pdfa% formato PDF/A, obbligatorio per l'archiviazione delle tesi di Polito
,cucitura%lascia margine per la rilegatura
%,twoside% per stampa fronte-retro (fortemente consigliato per tesi voluminose, opzionale per le altre)
%,12pt% font pi� grande (12pt) rispetto a quello normalmente usato (11pt)
]{toptesi}
%
\usepackage{hyperref}
\usepackage{comment}
\hypersetup{%
    pdfpagemode={UseOutlines},
    bookmarksopen,
    pdfstartview={FitH},
    colorlinks,
    linkcolor={blue},
    citecolor={red},
    urlcolor={blue}
  }


\usepackage[latin1]{inputenc}% IMPORTANTE! usare codifica ISO-8859-1 per le lettere accentate

\input{commands.tex}



\begin{document}
\selectlanguage{english}

\ateneo{Politecnico di Torino}

\titolo{Format Preserving Encryption for databases}

\corsodilaurea{Ingegneria Informatica}

\candidato{Francesco \textsc{Vaccaro}}


\relatore{prof.\ Antonio Lioy}
\secondorelatore{prof.\ Andrea Atzeni}

\tutoreaziendale{ing.\ Marco Mangiulli, ing.\ Luca Castello}


\sedutadilaurea{\textsc{mese} 20**}

\logosede{logopolito}


\errorcontextlines=9

\frontespizio
\paginavuota
\newpage
%per sfruttare meglio lo spazio nella pagina
\advance\voffset -5mm
\advance\textheight 30mm


% opzionale, solo se si vuole dedicare la tesi a delle persone care
\begin{dedica}
A mio padre

\textdagger\ A mio nonno Pino
\end{dedica}

\sommario

Inserire qui un breve sommario della tesi. (INSERITO QUELLO DELLA TRACCIA)

Companies migration to the cloud implies protection of their sensitive and private data. Encryption is the key tool for business's confidential data protection against cyber security threats. However, storing the data in an encrypted format requires to address critical issues: performance, format and ordering of data, protection of encryption keys. Format-preserving encryption (FPE) allows to encrypt the data in such a way that the output (the ciphertext) is in the same format as the input (the plaintext).

\ringraziamenti

Opzionali, solo nel caso si sia ricevuto un aiuto speciale e particolarmente rilevante.


\indici

\mainmatter

\chapter{State of the art}
%!TEX encoding = IsoLatin
%!TEX main = s288032.tex

\section {Sezione 1}


Iniziare a scrivere qualcosa di prova sulo State of the Art


\section {Sezione 2}


Continuare a scrivere qualcosa di prova


\selectlanguage{english}
\section{English Section}

Do we want to write the thesis in English?

\chapter{Analysis of the approved FPE mode}
%!TEX encoding = IsoLatin
%!TEX main = s288032.tex

\section {Sezione 1 - SP 800-38G}

\section {Spiegare cos'� un `tweak`, ( tweakable block cipher paper)}

\section {Sezione 2 - FF1}

FF1

\section {Sezione 3 - FF3}

FF3

\section {Sezione 4 - FF2}

FF2


\chapter{Proof of concept}
%!TEX encoding = IsoLatin
%!TEX main = s288032.tex

\section {Sezione 1}


Scrivere come � stata implementata la Proof of Concept


\section {Sezione 2}


Risultati Proof of Concept


%\selectlanguage{english}
\section{English Section}

Do we want to write the thesis in English?

\chapter{Risultati}

\chapter{Conclusioni}

% bibliografia scritta "a mano"
\input{biblio.tex}

% se la bibliografia � stata scritta (usando Bibtex) nel file biblio.bib allora commentare la riga precedente e scommentare le due righe seguenti
%\bibliographystyle{torsec}
%\bibliography{biblio}


\end{document}